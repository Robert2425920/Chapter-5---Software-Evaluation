

\chapter{Application Performance Summary\label{Appendix:PerformanceSummary}}





\section{Application Performance Summary}


\subsection{Criteria and Key}
This section shows the performance of each application across eight areas of functionality. The eight areas are:
\begin{enumerate}
    \item Document Handling
    \item Viewing Tools
    \item Editing Tools
    \item Files
    \item Screen
    \item Navigation, Back, Home, Global positioning
    \item Help Inline, Online, and Contextual
    \item Video Presentation
    
    
\end{enumerate}

\noindent In the tables below the following key is used:
\begin{itemize}
	\item[\color{green}\tick]\color{black} Positive functionality performance
    \item [\color{amber}!!]\color{black} Needs attention for stability
     \item [\color{red}\cross]\color{black} Critical issue that must be resolved.
     \item For noting.
\end{itemize}
%need to put key to tables bullet points in here

\newpage
\subsection{Document Handling}
%\singlespace


\begin{center}
\begin{table}[htbp]
\label{tab:Document Handling}    
\caption{Page Turning}
%\begin{tabular}{ |p{3cm}|p{3cm}|p{3cm}|  }
\small
\centering
\begin{tabular}{|p{0.15\textwidth}|p{0.7\textwidth}|}
\hline
\rowcolor{lightgrey}\multicolumn{2}{|c|}{Page Turning}\\
\hline
Exhibit A & \begin{itemize}
    \item To change pages on a PDF use the arrows on the side of the page - poor visibility
\end{itemize}\\
\hline
TrialPad & \begin{itemize}
    \item To change pages on a PDF use arrows at the bottom of the screen. Able to flick pages up and down
\end{itemize}\\
\hline
RLTC & \begin{itemize}
    \item To change pages of a PDF use arrows at the top of the page.This is actually what appears in the instructions, however could not get arrows to appear.
 
\end{itemize}\\
\hline
\end{tabular}

\end{table}
\end{center}

\begin{center}
\begin{table}[h!]
\label{tab:PageNavigation}    
\caption{Page Navigation}
\centering
\begin{tabular}{|p{0.15\textwidth}|p{0.7\textwidth}|}

\rowcolor{lightgrey}\multicolumn{2}{|c|}{Page Navigation}\\
\hline
Exhibit A & \begin{itemize}
    \item Slide bar on the side of the PDF shows where you are upto in the file.
\end{itemize}\\
\hline
TrialPad & \begin{itemize}
    \item Slide bar on the side of the PDF shows where you are upto in the file.
\end{itemize}\\
\hline
RLTC & \begin{itemize}
    \item Current page number out of total on slide informs of position in document. No slide bar.
\end{itemize}\\
\hline
\end{tabular}


\end{table}
\end{center}

\newpage

\subsection{Viewing Tools}

%\definecolor{lightgrey}{RGB}{200,205,207}
%\begin{tabular}{ |p{3cm}|p{3cm}|p{3cm}|  }
\begin{center}
\begin{table}[htbp]

\label{tab:Magnify}    
\caption{Magnify}
\centering
\begin{tabular}{|p{0.15\textwidth}|p{0.7\textwidth}|}
\hline
\rowcolor{lightgrey}\multicolumn{2}{|c|}{Magnify}\\
\hline
Exhibit A &
 \begin{itemize}
    \item[\color{amber}!!]\color{black} When using Magnify the clear or undo functionality is not available. Have to go back to the Magnify Glass to reduce.
    \item Very grainy when magnified.
 \end{itemize}\\
\hline
TrialPad &
 \begin{itemize}
    \item[\color{amber}!!]\color{black} When using Magnify you have to use the clear button to remove the magnification, the undo button does not work.
     \item[\color{green}\tick]\color{black}Magnified image is very clear.
 \end{itemize}\\
\hline
RLTC &
 \begin{itemize}
    \item Pressing on an area of the document magnifies the area. Double tapping magnifies the page.
\end{itemize}\\
\hline
\end{tabular}
\end{table}
\end {center}





\begin{center}
\begin{table}[htbp]

\label{tab:Rotation}    
\caption{Rotation}
\centering
\begin{tabular}{|p{0.15\textwidth}|p{0.7\textwidth}|}
\hline
\rowcolor{lightgrey}\multicolumn{2}{|c|}{Rotation}\\
\hline
Exhibit A &
 \begin{itemize}
    \item Rotation gives 4 directions from one button.
    \item Quick to view and select.
\end{itemize}\\
\hline
TrialPad &
 \begin{itemize}
    \item Able to rotate page or document.
    \item[\color{amber}!!]\color{black} Rotation is uni-directional.
\end{itemize}\\
\hline
RLTC &
 \begin{itemize}
    \item[\color{red}\cross]\color{black} Rotation is not available.
   
\end{itemize}\\
\hline
\end{tabular}\\
\end{table}
\end{center}
%table: Laser pointer

\begin{center}
\begin{table}[htbp]

\label{tab:LaserPointer}    
\caption{Laser Pointer}
\centering
\begin{tabular}{|p{0.15\textwidth}|p{0.7\textwidth}|}
\hline
\rowcolor{lightgrey}\multicolumn{2}{|c|}{Laser Pointer}\\
\hline
Exhibit A  &
 \begin{itemize}
    \item[\color{green}\tick]\color{black} Easy to use, works well.
    \item[\color{amber}!!]\color{black} Tool used with finger on screen. Accuracy is an issue.
   
\end{itemize}\\
\hline
TrialPad &
 \begin{itemize}
    \item[\color{green}\tick]\color{black} Easy to use, works well.
    \item[\color{amber}!!]\color{black} Tool used with finger on screen. Accuracy is an issue.
   
\end{itemize}\\
\hline
RLTC &
 \begin{itemize}
    \item[\color{red}\cross]\color{black} Laser pointer not available.
\end{itemize}\\
\hline
\end{tabular}
\end{table}
\end{center}



\newpage
\subsection{Editing Tools}



\begin{center}
\begin{table}[htbp]

\label{tab:Highlight}    
\caption{Highlight}
\centering
\begin{tabular}{|p{0.15\textwidth}|p{0.7\textwidth}|}
\hline
\rowcolor{lightgrey}\multicolumn{2}{|c|}{Highlight}\\
\hline
Exhibit A &
 \begin{itemize}
    \item[\color{green}\tick]\color{black} 2 highlight modes available: Freehand and Block.
    \item[\color{amber}!!]\color{black} Tool used with finger on screen. Accuracy is an issue.
    \item Highligh available under tools
\end{itemize}\\
\hline
TrialPad &
 \begin{itemize}
    \item[\color{green}\tick]\color{black} Highlight pen is very clear and easy to use.
    \item[\color{amber}!!]\color{black} Tool used with finger on screen. Accuracy is an issue.
    \item[\color{amber}!!]\color{black} Highlight only in block mode, no freehand.
\end{itemize}\\
\hline
RLTC &
 \begin{itemize}
    \item Highlight pen is clear and easy to use.
    \item[\color{amber}!!]\color{black} Tool used with finger on screen. Accuracy is an issue.
    \item[\color{amber}!!]\color{black} Highlight only in block mode, no freehand.
    \item[\color{red}\cross]\color{black} After tapping the highlight it was not possible to remove highlight mode. The clear works to remove what is on screen, however the highlight mode remains on. This is a major issue
\end{itemize}\\
\hline

\end{tabular}\\
\end{table}
\end{center}

\begin{center}
\begin{table}[htbp]

\label{tab:PencilLine}    
\caption{Pencil Line}
\centering
\begin{tabular}{|p{0.15\textwidth}|p{0.7\textwidth}|}
\hline
\rowcolor{lightgrey}\multicolumn{2}{|c|}{Pencil Line}\\
\hline
Exhibit A &
 \begin{itemize}
	\item 5 colored pencils with changable width.
	\item Clean smooth but does not recognise white space for underlining.
 \end{itemize}\\
\hline
TrialPad & 
 \begin{itemize}
	\item 4 colored pencils with changable width.
	\item [\color{amber}!!]\color{black}Extra features available (hold down for menu), but not obvious to new users.
	\item [\color{amber}!!]\color{black}Can not select color and width at the same time, need to close and reopen tool to do each action.
 \end{itemize}\\
\hline
RLTC & 
\begin{itemize}
  \item [\color{amber}!!]\color{black} Not available.
\end{itemize}\\
\hline

\end{tabular}
\end{table}
\end{center}

\begin{center}
\begin{table}[htbp]

\label{tab:PostItNotes}    
\caption{Post It Notes}
\centering
\begin{tabular}{|p{0.15\textwidth}|p{0.7\textwidth}|}
\hline
\rowcolor{lightgrey}\multicolumn{2}{|c|}{Post It Notes}\\
\hline
Exhibit A & Could use redact in 5 colours, as redact block able to be annotated using the pencil.\\
\hline
TrialPad & [\color{amber}!!]\color{black} Not available.\\
\hline
RLTC & [\color{amber}!!]\color{black} Not available.\\
\hline
\end{tabular}
\end{table}
\end{center}

\begin{center}
\begin{table}[htbp]

\label{tab:Redact}    
\caption{Redact}
\centering
\begin{tabular}{|p{0.15\textwidth}|p{0.7\textwidth}|}
\hline
\rowcolor{lightgrey}\multicolumn{2}{|c|}{Redact}\\
\hline
Exhibit A & 
\begin{itemize}
	\item Five colours available.
	\item Can be written on with freehand pencil.
	\item Could be used for Post It notes, but would obscure part of the screen.
\end{itemize}\\
\hline
TrialPad &
\begin{itemize}
  \item Only available black and white .
  \item Can write over the top to make notes.
\end{itemize}\\
\hline
RLTC & 
\begin{itemize}
  \item [\color{amber}!!]\color{black}Not available
\end{itemize}\\
\hline

\end{tabular}
\end{table}
\end{center}

\begin{center}
\begin{table}[htbp]

\label{tab:OtherFeatures}    
\caption{Other Features}
\centering
\begin{tabular}{|p{0.15\textwidth}|p{0.7\textwidth}|}
\hline
\rowcolor{lightgrey}\multicolumn{2}{|c|}{Other Features}\\
\hline
RLTC & 
\begin{itemize}
  \item Press on word enables you to copy or define word.
  \item [\color{red}\cross]\color{black} Can't work out how to paste.
\end{itemize}\\
\hline
\end{tabular}
\end{table}
\end{center}

\newpage
\subsection{Files}

\begin{center}

\begin{table}[htbp]
\label{tab:FileTypesSupported}%
  \centering
  \caption{File Types supported}
    \begin{tabular}{|p{0.15\textwidth}|p{0.7\textwidth}|}
    \hline
    \rowcolor{lightgrey}\multicolumn{2}{|c|}{File types supported}\\
    \hline
%     \multirow{2}[4]{*}{Exhibit A} & PDF, images, and video \\
% 	\cline{2-2}          
% 	& Limited support for: docs, .txt files , excel, powerpoint \\
	Exhibit A &
	\begin{itemize}
	  \item PDF, images, and video
	  \item Limited support for: docs, .txt files , excel, powerpoint
	\end{itemize}\\
    \hline
    TrialPad & 
    \begin{itemize}
      \item PDF, docs, excel, keynote, video
    \end{itemize} \\
    \hline
    RLTC  & 
    \begin{itemize}
      \item PDF and image only
    \end{itemize} \\
    \hline
    \end{tabular}
  
\end{table}
\end{center}


\begin{center}

\begin{table}[htbp]
\label{tab:DownloadMethod}
  \centering
  \caption{Download Methods}
    \begin{tabular}{|p{0.15\textwidth}|p{0.7\textwidth}|}
    \hline    
    \rowcolor{lightgrey}\multicolumn{2}{|c|}{Download Methods}\\
    \hline
    Exhibit A & 
    \begin{itemize}
      \item Documents, photos, and video from Dropbox, email, ftp, iPhoto on iPad 
      \item Photos and video from iTunes or Dropbox
    \end{itemize} \\
    \hline
    TrialPad & 
    \begin{itemize}
      \item Documents, photos, and video from Dropbox, email, iPhoto on iPad
      \item Photos and video from iTunes or Dropbox
      \item [\color{amber}!!]\color{black}Need physical link to computer with iTunes
    \end{itemize} \\
    \hline


    RLTC  & 
    \begin{itemize}
      \item [\color{amber}!!]\color{black}Itunes only
      \item [\color{amber}!!]\color{black}Single file download only
    \end{itemize} \\
    \hline
    \end{tabular}%
  
\end{table}%
\end{center}


\begin{center}

\begin{table}[htbp]
  \centering
  \caption{Archive \& Dropbox Support}
    \begin{tabular}{|p{0.15\textwidth}|p{0.7\textwidth}|}
    \hline
    \rowcolor{lightgrey}\multicolumn{2}{|c|}{Archive \& Dropbox Support}\\
    \hline
    Exhibit A & 
    \begin{itemize}
      \item [\color{amber}!!]\color{black}Linking to Dropbox. When you select unlink from Dropbox, the cautionary notice only confirms thet you have unlinked and does not offer the oppurtunity to cancel the command.
      \item [\color{amber}!!]\color{black}Only able to download one file at a time, not a folder. Claims to but could not get it to work.
      \item No way to unzip file of evidence to overcome single file download.
    \end{itemize}\\
	\hline
    TrialPad & 
    \begin{itemize}
      \item [\color{amber}!!]\color{black}Linking to Dropbox. When you select unlink there is no cautionary notice. Wrong selection will cause down time and confusion.
      \item [\color{green}\tick]\color{black}Able to download files and folders.
      \item No way to unzip a file, howerer a folder download may suffice.
    \end{itemize}\\
    \hline
    RLTC  & 
    \begin{itemize}
      \item [\color{amber}!!]\color{black}Not available.
    \end{itemize} \\
    \hline
    \end{tabular}%
  \label{tab:SaveToFolder}%
\end{table}
\end{center}

\begin{center}
% Table generated by Excel2LaTeX from sheet 'Sheet1'
\begin{table}[htbp]
  \centering
  \caption{Save to Folder}
    \begin{tabular}{|p{0.15\textwidth}|p{0.7\textwidth}|}
    \hline
    \rowcolor{lightgrey}\multicolumn{2}{|c|}{Save to Folder}\\
    Exhibit A & 
    \begin{itemize}
      \item [\color{amber}!!]\color{black}No indication of where a file is saved, or that it has been saved. Poor system visability.
      \item Saved file actually goes to the case project, which is where it should be.
    \end{itemize} \\
    \hline
    TrialPad & 
    \begin{itemize}
      \item Able to select where to save download files.
    \end{itemize} \\
    \hline
    RLTC  & 
    \begin{itemize}
      \item [\color{amber}!!]\color{black}All files download to a single file structure.
    \end{itemize} \\
    \hline
    \end{tabular}%
  \label{tab:SaveToExternal}%
\end{table}%
\end{center}

\begin{center}

\begin{table}[htbp]
  \centering
  \caption{Save to External}
    \begin{tabular}{|p{0.15\textwidth}|p{0.7\textwidth}|}
    \hline
    \rowcolor{lightgrey}\multicolumn{2}{|c|}{Save to External}\\
    \hline
    Exhibit A & 
    \begin{itemize}
      \item Files can be uploaded to Dropbox.
    \end{itemize}\\
    \hline
    TrialPad & 
    \begin{itemize}
      \item To save a file there is the choice of email, print, or Dropbox for either the page or the file. 
      \item Icon used is the iPhone Send.
    \end{itemize}\\
    \hline
    RLTC  & 
    \begin{itemize}
      \item [\color{red}\cross]\color{black}No save function.
    \end{itemize} \\
    \hline
    \end{tabular}%
  \label{tab:FileManagement}%
\end{table}
\end{center}

\begin{center}
% Table generated by Excel2LaTeX from sheet 'Sheet1'
\begin{table}[htbp]
  \centering
  \caption{File Management}
    \begin{tabular}{|p{0.15\textwidth}|p{0.7\textwidth}|}
    \rowcolor{lightgrey}\multicolumn{2}{|c|}{File Management} \\
    \hline
    Exhibit A & 
    \begin{itemize}
      \item [\color{amber}!!]\color{black}Use the ``my projects'' to describe cases.
      \item [\color{green}\tick]\color{black}Uses folder structure within projects as well as files.
    \end{itemize} \\
    \hline
    TrialPad & 
    \begin{itemize}
      \item [\color{green}\tick]\color{black}Refer to ``cases'' for the cases folder. Obvious how to name and describe.
      \item [\color{amber}!!]\color{black}To delete a folder/document select the edit button, then select the folder / file then select trash. This is a lot of steps and no cautionary notice.
      \item [\color{amber}!!]\color{black}Same method for duplicate, rename, and move.
      \item Store files and folders between document and video / key docs for easy referal.
      \item [\color{amber}!!]\color{black}Images in documents. Why not in images tab?
      \item [\color{green}\tick]\color{black}At the file level in the file manager the add button still relates to the new folder. Good consistancy.
    \end{itemize} \\
    \hline
    RLTC  &
    \begin{itemize}
      \item [\color{red}\cross]\color{black}No folder or file structure.
      \item [\color{red}\cross]\color{black}Not able to delete documents.
      \item [\color{amber}!!]\color{black}Does not sort file list.
    \end{itemize}  \\
    \hline
    \end{tabular}%
 
\end{table}

\end{center}



\begin{center}
% Table generated by Excel2LaTeX from sheet 'Sheet1'
\begin{table}[htbp]
\label{tab:CreateNewTable}%
  \centering
  \caption{Create New Case and Folder}
    \begin{tabular}{|p{0.15\textwidth}|p{0.7\textwidth}|}
    \hline
    \rowcolor{lightgrey}\multicolumn{2}{c}{Create New Case and Folder} \\
    Exhibit A & \begin{itemize}
      \item [\color{amber}!!]\color{black}To create a new case use the + case icon on the Cases screen. The + case icon appears at the end of the of the cases screen, and is not visible if more than two cases are already in the system.
      \item To add a folder use ``add folder''. 
	  \item To add an exhibit use ``add exhibit''.
    \end{itemize}\\
    \hline
    TrialPad & \begin{itemize}
      \item [\color{green}\tick]\color{black}To create a new case use the + button. This is consistant with adding files.
      \item [\color{amber}!!]\color{black}Tools to create a new case or add a file are in the ``shadow'' of the case. They are small and pale.
    \end{itemize}\\
    \hline
    RLTC  & 
    \begin{itemize}
      \item [\color{red}\cross]\color{black}No cases or folders available.
    \end{itemize} \\
    \hline
    \end{tabular}%
  
\end{table}
\end{center}


\begin{center}
% Table generated by Excel2LaTeX from sheet 'Sheet1'
\begin{table}[htbp]
\label{tab:Search}
  \centering
  \caption{Search}
    \begin{tabular}{|p{0.15\textwidth}|p{0.7\textwidth}|}
    \hline
    \rowcolor{lightgrey}\multicolumn{2}{c}{Search} \\
    \hline
    Exhibit A &
    \begin{itemize}
      \item Searchbox using search icon.
      \item [\color{green}\tick]\color{black}Will do search on partial entry of filename, but not folder.
    \end{itemize}\\
    \hline
    TrialPad &
    \begin{itemize}
      \item 2 search boxes. One for Case on opening page and one for folders and files abole file listing.
      \item [\color{green}\tick]\color{black}Full name and partial name search available.
    \end{itemize}\\
    \hline
    RLTC &
    \begin{itemize}
      \item [\color{red}\cross]\color{black}Not available.
    \end{itemize}\\
    \hline
\end{tabular}
\end{table}
\end{center}

\newpage
\subsection{Screen}
\begin{center}
% Table generated by Excel2LaTeX from sheet 'Sheet1'
\begin{table}[htbp]
\label{tab:Orientation}
  \centering
  \caption{Orientation}
    \begin{tabular}{|p{0.15\textwidth}|p{0.7\textwidth}|}
    \hline
    \rowcolor{lightgrey}\multicolumn{2}{c}{Orientation} \\
    \hline
    Exhibit A &
    \begin{itemize}
      \item Landscape only.
      
    \end{itemize}\\
    \hline
    TrialPad &
    \begin{itemize}
      \item Landscape and Portriat.
      
    \end{itemize}\\
    \hline
    RLTC &
    \begin{itemize}
      \item Landscape only.
    \end{itemize}\\
    \hline
\end{tabular}
\end{table}
\end{center}


\begin{center}
% Table generated by Excel2LaTeX from sheet 'Sheet1'
\begin{table}[htbp]
\label{tab:UsageOfAvailableSpace}
  \centering
  \caption{Usage of Available Space}
    \begin{tabular}{|p{0.15\textwidth}|p{0.7\textwidth}|}
    \hline
    \rowcolor{lightgrey}\multicolumn{2}{c}{Usage of Available Space} \\
    \hline
    Exhibit A &
    \begin{itemize}
      \item Can hide toolbar for greater screen space.
    \end{itemize}\\
    \hline
    TrialPad &
    \begin{itemize}
      \item Cannot hide toolbar for greater screen space.
     \item [\color{amber}!!]\color{black}Available file structure are displayed on screen at all times, using too much screen space in Landscape view. Better to have a drop-down list that reduces when user is working with the document. Not an issue with projection.
      \item In portrait mode screen view loses file structure, giving all screen space to the evidence.
    \end{itemize}\\
    \hline
    RLTC &
    \begin{itemize}
      \item Cannot hide toolbar for greater screen space.
         \end{itemize}\\
    \hline
\end{tabular}
\end{table}
\end{center}

\newpage
\subsection{Navigation: Back, Home, Global positioning}

\begin{center}
% Table generated by Excel2LaTeX from sheet 'Sheet1'
\begin{table}[htbp]
\label{tab:Navigation}
  \centering
  \caption{Navigation}
    \begin{tabular}{|p{0.15\textwidth}|p{0.7\textwidth}|}
    \hline
    \rowcolor{lightgrey}\multicolumn{2}{c}{Navigation} \\
    \hline
    Exhibit A &
    \begin{itemize}
      \item Back key only available at certain stages. For example not available going from inside a case back to home. The home button is also not available; the choice has to be my projects. Lack of consistancy in how this is handled.
    \end{itemize}\\
    \hline
    TrialPad &
    \begin{itemize}
      \item [\color{amber}!!]\color{black}Return key only gives previous directory, which does not show where you are in the system. May cause navigation problems.
     \item [\color{amber}!!]\color{black}No Home key available.
     
    \end{itemize}\\
    \hline
    RLTC &
    \begin{itemize}
      \item No navigation, but none necessary as there is no file structure. 
         \end{itemize}\\
    \hline
\end{tabular}
\end{table}
\end{center}


\newpage
\subsection{Help: Inline, Online, and Contextual}

\begin{center}
% Table generated by Excel2LaTeX from sheet 'Sheet1'
\begin{table}[htbp]
\label{tab:Help}
  \centering
  \caption{Help: Inline, Online, and Contextual}
    \begin{tabular}{|p{0.15\textwidth}|p{0.7\textwidth}|}
    \hline
    \rowcolor{lightgrey}\multicolumn{2}{c}{Help: Inline, Online, and Contextual} \\
    \hline
    Exhibit A &
    \begin{itemize}
      \item First help on the bottom of the screen relates to download methods (Dropbox ect.) with a contact us available.
      \item Second help function (i-help) top right of screen is to let users know there are short cuts to add/edit exhibits and folders. The facility to tap for further information is not obvious, hence the need for the i-help button.
      \end{itemize}\\
    \hline
    TrialPad &
    \begin{itemize}
      \item [\color{amber}!!]\color{black}Help is a link to a website,so necessary to have internet connection. Basic help and contextual hel at the point of the problem should be available.
       \item Short tutorial videos on the website are useful for learning, but not for trouble shooting.
     
    \end{itemize}\\
    \hline
    RLTC &
    \begin{itemize}
      \item [\color{red}\cross]\color{black}Help is a one page description of basics. Cannot find any support of value on the web. 
      \item No contextual help.  
    \end{itemize}\\
    \hline
\end{tabular}
\end{table}
\end{center}


% \newpage
\subsection{Video Presentation}

\begin{center}
% Table generated by Excel2LaTeX from sheet 'Sheet1'
\begin{table}[htbp]
\label{tab:Video}
  \centering
  \caption{Video Presentation}
    \begin{tabular}{|p{0.15\textwidth}|p{0.7\textwidth}|}
    \hline
    \rowcolor{lightgrey}\multicolumn{2}{c}{Video Presentation} \\
    \hline
    Exhibit A &
    \begin{itemize}
      \item Video support is listed but could not get to work
    \end{itemize}\\
    \hline
    TrialPad &
    \begin{itemize}
      \item Perpetual control bar shown over video blocks the view of the video on the screen. The control bar is nor shown on the projected view.
      \item [\color{green}\tick]\color{black}Can do quick screen capture function, and two stage zooming.
      \item [\color{green}\tick]\color{black}Can create a clip of the video that is stored in the file manager.
     
    \end{itemize}\\
    \hline
    RLTC &
    \begin{itemize}
      \item [\color{red}\cross]\color{black} No video support  
    \end{itemize}\\
    \hline
\end{tabular}
\end{table}
\end{center}

\subsection{Note}
* PDF annotator has a snap and fit straight line underliner that starts at the first point of contact to underline or cross out if the initial contact point is in the middle of the text. These packages simply draw as crookedly as the user does.\\

** Projection: Each of the packages use different terminology for presenting the screen.
\begin{itemize}
  \item TrialPad offers variations of how the screen could be viewed using output on and off, and a blank, freeze, and present as options for presentation. 
  \item Exhibit A only offers ``show'' as a choice for project/not project.
  \item RLTC Evidence used ``publish''which required selection after any changes were made such as highlight/project, underline/project ect.
\end{itemize}

