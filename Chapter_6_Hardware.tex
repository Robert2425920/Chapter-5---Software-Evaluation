\chapter{Hardware\label{chap:Hardware}}

%Introduction ===============================================================================
\section{Introduction\label{Section:Hardware:Introduction}}

Evidence presented in Australian courtrooms is increasing in technical complexity continually. This increase in the technical complexity of evidence is largely due to avdancements in scientific evidence collection techniques. As a result of this increasing complexity of evidence the requirements for presentation for this evidence has also become far more complex. To date, the solution to the display needs for evidence is approached in a adhock way based on the singular type of evidence being presented. This has resulted in a patchwork of technologies being used with little or no integration between the various technologies used for presentation, and very little standardisation of technologies between the different Courts, and between different courtrooms within a Court. 

Introduction of an integrated evidence display system for Australian courtrooms is very complex. Not only does the software need the correct functioanlity, and excellent usability (discussed in  \ref{chap:Software}), but it also needs the correct choice of hardware. The rapid adoption of tablet technology seen not only in the Australian population, but also within the Australian judiciary \ref{Heerboth2013}can be seen as a very real opportunity to advance evidence presentation by using this technology.

For the introduction of new evidence display technology it is essential to understand how the stakeholders will perceive its introduction. The Technology Acceptance Model (TAM) identified two key areas for technology acceptance. They are precieved usefulness of the technology, and the percieved ease of use of the technology. To accurately guage these two areas for evidence display technology the involvement and input from all members of the courtroom is critical. 

This chapter will explore two studies conducted with key stakeholders designed to guage the usefullness, and the ease of use of tablet technology for evidence presentation purposes. The first study was conducted as a future technology workshop, allowing candidates to operate iPads for the purpose of displaying evidence. The second study, informed by the results of the first study, was a Moot Court session using centralised control and push technology to allow the candidates to participate in a real trial situation. Each study was conducted twice with different canidiates.

The studies will be examined in detail, results analysed, and recommendations made.  





% Studies & Methodologies====================================================================
\section{Studies \& Methodologies\label{Hardware:Studies and Methodologies}}








% Study 1 ===================================================================================
\section{Study 1\label{Hardware:Study 1}}
\subsection{Study Design}


\subsection{Results}



\subsection{Analysis and Recommendations}







% Study 2=====================================================================================
\section{Study 2\label{Hardware:Study 2}}













% Discussion ================================================================================
\section{Discussion\label{Hardware:Discussion}}









% Closing Comments ==========================================================================
\Section{Closing Comments\label{Hardware:Closing Comments}}