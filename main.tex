\documentclass{article}
\usepackage[utf8]{inputenc}
 \usepackage{setspace}
 \usepackage{color}

\title{Chapter 5 - Software Evaluation}
\author{rtipping }
\date{September 2018}

\begin{document}

\maketitle
\tableofcontents
\section{Introduction}
\doublespace
Technology that is used for evidence presentation in Australian courtrooms must support all of the required legal process of the Courts. While there are many issues relating to the hardware requirements (discussed in later chapters), the specific requirements of the Software used must also be thoroughly examined. The uptake of tablet technology in society, and more specifically with the legal system is accelerating. Barristers and judges are starting to see the advantage of placing their court notes on iPads, allowing them easy access from within the courtroom. iPads are also being used by magistrates and judges for many other court related tasks. The increasing level of acceptance of tablet technology offers the opportunity to consider using tablet technology for evidence presentation purpose in Australian courtrooms.
In this chapter these Software requirements for applications to be used for evidence presentation will be examined, leading evidence presentation software packages for the iPad will be compared, and recommendations for required functionality will be made.
The three applications that will be compared are Exhibit A v 1.5 by Lectura LLC ,  TrialPad v2.1 by Lit Software LLC, RLTC Evidence v 1.1 by Rosen LTC.


\section{Method}
This study was conducted in three phases, requirements gathering, application usability testing, and analysis of the applications against identified requirements. Techniques employed in this study were from the discipline of Human Computer Interaction (HCI) \color{green}(Preece et al., 2007)
\color{black}
\subsection{Requirements Gathering}
Establishing the requirements for this study was conducted in two phases. The first phase involved Field Studies. This phase was conducted over a two month period observing evidence being presented in various courtrooms, including the Magistrates Court, the County Court, and the Supreme Court. While conducting these field studies informal discussions were held with Judges, magistrates, barristers, expert witnesses, tipstaff, and court reporters. \color{green}(is this the correct term?\color{black}\\
An affinity diagram was then created for identified behaviours, activities, tasks, tools and goals. These were categorised into physical, organisational, technical, and social contexts.
\subsection{Usability Application Analysis}
Three leading iPad applications for evidence presentation that are being used in the American judicial system were used as a starting point for this phase of the study. The three applications that were investigated are Exhibit A v 1.5 by Lectura LLC ,  TrialPad v2.1 by Lit Software LLC, RLTC Evidence v 1.1 by Rosen LTC. 
Using the requirements established in phase one of this study the three applications were evaluated and tested to establish their suitability for use in Australian courtrooms, with a focus on each applications included functionality, and overall usability. Particular attention was given to the functionality, and establishing if each application actually did what it claimed to do.\\
Phase one of the application analysis was conducted in a usability laboratory. The laboratory environment allowed for testing to focus solely on the application interface without interference and distractions from other environmental factors.
The purpose of this test was to discover any usability issues with the applications, and to establish their alignment with the already ascertained requirements. The test candidates used for this study were HCI experts. HCI experts were used as they already had the required skill set to identify any usability issues that they encountered in testing the application. A total of 24 testers were used.None of the testers had any prior experience with any of the applications being tested, and most had only limited experience with tablet technologies.\\
Each of the testers were asked to perform a number of tasks that had been identified as common or critical tasks for the use of the applications in Australian courtrooms. No stylus or pointing devices were used throughout the tests, and test candidates only used their fingers to interact with the applications. Candidates used voice protocols to allow for capture of opinion and activity during each test. Dropbox \copyright\   was used during this phase of the study for the purpose of downloading the  evidence files onto the iPads.
\subsection{Moot Court}
Phase two of the application testing was conducted in a Moot Court environment. Nine test candidates were used in this phase of the testing. None of the candidates had any previous experience with any of the applications, and only limited experience using tablet technologies. The candidates had been involved in the courtroom visits, so had some understanding of the courtroom environment, courtroom processes, and behaviours. The test were conducted using a scripted scenario that required the candidates to present evidence in a mock trial. Candidates played the roles of Magistrate, Defence Council, and Prosecuting Council. No styluses, or pointing devices were used throughout this test, candidates interacted with the applications using only their fingers. The iPads used in the test were connected to a data projector via a physical cable to allow for the interface to be closely monitored throughout the test. This physical cable connection did not impede the candidates use of the iPads.

\section{Results}


\section{Closing Comments}

\end{document}
