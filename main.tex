\documentclass{article}
\usepackage[utf8]{inputenc}
 \usepackage{setspace}
 \usepackage{color}

\title{Chapter 5 - Software Evaluation}
\author{rtipping }
\date{September 2018}

\begin{document}

\maketitle
\tableofcontents
\section{Introduction}
\doublespace
Technology that is used for evidence presentation in Australian courtrooms must support all of the required legal process of the Courts. While there are many issues relating to the hardware requirements (discussed in later chapters), the specific requirements of the Software used must also be thoroughly examined. The uptake of tablet technology in society, and more specifically with the legal system is accelerating. Barristers and judges are starting to see the advantage of placing their court notes on iPads, allowing them easy access from within the courtroom. iPads are also being used by magistrates and judges for many other court related tasks. The increasing level of acceptance of tablet technology offers the opportunity to consider using tablet technology for evidence presentation purpose in Australian courtrooms.
In this chapter these Software requirements for applications to be used for evidence presentation will be examined, leading evidence presentation software packages for the iPad will be compared, and recommendations for required functionality will be made.
The three applications that will be compared are Exhibit A v 1.5 by Lectura LLC ,  TrialPad v2.1 by Lit Software LLC, RLTC Evidence v 1.1 by Rosen LTC.


\section{Method}
This study was conducted in three phases, requirements gathering, application usability testing, and analysis of the applications against identified requirements. Techniques employed in this study were from the discipline of Human Computer Interaction (HCI) \color{green}(Preece et al., 2007)
\color{black}
\subsection{Requirements Gathering}
Establishing the requirements for this study was conducted in two phases. The first phase involved Field Studies. This phase was conducted over a two month period observing evidence being presented in various courtrooms, including the Magistrates Court, the County Court, and the Supreme Court. While conducting these field studies informal discussions were held with Judges, magistrates, barristers, expert witnesses, tipstaff, and court reporters. \color{green}(is this the correct term?\color{black}\\
An affinity diagram was then created for identified behaviours, activities, tasks, tools and goals. These were categorised into physical, organisational, technical, and social contexts.
\subsection{Usability Application Analysis}
Three leading iPad applications for evidence presentation that are being used in the American judicial system were used as a starting point for this phase of the study. The three applications that were investigated are Exhibit A v 1.5 by Lectura LLC ,  TrialPad v2.1 by Lit Software LLC, RLTC Evidence v 1.1 by Rosen LTC. 
Using the requirements established in phase one of this study the three applications were evaluated and tested to establish their suitability for use in Australian courtrooms, with a focus on each applications included functionality, and overall usability. Particular attention was given to the functionality, and establishing if each application actually did what it claimed to do.\\
Phase one of the application analysis was conducted in a usability laboratory. The laboratory environment allowed for testing to focus solely on the application interface without interference and distractions from other environmental factors.
The purpose of this test was to discover any usability issues with the applications, and to establish their alignment with the already ascertained requirements. The test candidates used for this study were HCI experts. HCI experts were used as they already had the required skill set to identify any usability issues that they encountered in testing the application. A total of 24 testers were used.None of the testers had any prior experience with any of the applications being tested, and most had only limited experience with tablet technologies.\\
Each of the testers were asked to perform a number of tasks that had been identified as common or critical tasks for the use of the applications in Australian courtrooms. No stylus or pointing devices were used throughout the tests, and test candidates only used their fingers to interact with the applications. Candidates used voice protocols to allow for capture of opinion and activity during each test. Dropbox \copyright\   was used during this phase of the study for the purpose of downloading the  evidence files onto the iPads.
\subsection{Moot Court}
Phase two of the application testing was conducted in a Moot Court environment. Nine test candidates were used in this phase of the testing. None of the candidates had any previous experience with any of the applications, and only limited experience using tablet technologies. The candidates had been involved in the courtroom visits, so had some understanding of the courtroom environment, courtroom processes, and behaviours. The test were conducted using a scripted scenario that required the candidates to present evidence in a mock trial. Candidates played the roles of Magistrate, Defence Council, and Prosecuting Council. No styluses, or pointing devices were used throughout this test, candidates interacted with the applications using only their fingers. The iPads used in the test were connected to a data projector via a physical cable to allow for the interface to be closely monitored throughout the test. This physical cable connection did not impede the candidates use of the iPads.

\section{Results}
The requirements analysis identified many critical issues with current courtroom procedures that would benefit from an IT solution. These issues can be categorised into six main areas:evidence loading, navigation, file handling, evidence annotation, evidence display manipulation, and evidence display
\subsection{Issue 1: File Handling}
The large volume of evidence now required in many cases was clearly identified in the initial phase of this study. Current practice involves paper documents, that each have to be created, collated, and photocopied. Testers observed one case where several trolleys full of document files were wheeled into the courtroom. This is an extremely cumbersome process, and often results in delays and adjournments in cases while the correct documentary evidence is located, copied, and distributed to the various parties within the courtroom.\\
Due to the volume of evidence now required in many cases the method of loading evidence onto the tablets is of greater importance. A streamlined process to achieve evidence loading is critical for the usability of any system that will be developed.\\
\textit{iPad app as a solution:}\\
Loading of evidence files onto the test iPads was done via file transfer over a wireless network. Loading of the files was a very time consuming and cumbersome process. This issue will clearly be magnified in cases that involve larger volumes of evidence.\\
\textit{Test Observations:}\\
It was observed during the usability application analysis that the wireless network connection failed. The result of this failure rendered candidates unable to complete the loading of evidence from the Dropbox repository.\\
Evidence loading via a wireless network connection allows great flexibility for the application, but the quality of the wireless connection must be assured. The addition of a "hard wired" alternative would be of great benefit, both in terms of reliability, and also in file transfer rates. This hard wired alternative could take the form of a network connection via ethernet cable or the use of a memorystick via a usb port. iPads do not currently have an eithernet port or a usb port, other tablets do offer this.\\
\subsection{Issue 2: Navigation}
Navigation within documents is a critical issue. This issue is lessened when the documents are moved through in a sequential order in a case. During testing members of the court were often observed back through documents to find relevant and related material. Lengthy documents were of particular concern as this involved turning pages back and forth to find the relevant material within the same document.\\
\textit{iPad as a solution:}\\
Applications tested on the iPad offer scrolling within the document, and also a back button that allows the user to retrace their path through document files.\\
\textit{Test Observations:}
Test candidates were observed to often struggle navigating between the various function of the evidence presentation application. The lack of consistency within applications was identified as a major cause. An example of this is the "back" button in the Exhibit A application. The button was always in the same location on the screen, but depending on the page being displayed the text on the button would change. Test candidates often struggled to navigate to their desired location within the applications. One candidate was observed to actually give up on a task because of the frustration this caused. \\
The size of the buttons on the application interface was also observed to be an issue. Buttons were located very close together due to the limited work area offered by the iPad. It was observed during both the usability application analysis, and the moot court session that the users often selected the wrong button. This resulted not only in failing to navigate to the desired location, but in one case an evidence file being mistakenly annotated when the user mistakenly selected the pen button thinking they had selected a different functionality. When user are only using their fingers to interact with the tablet button size must be increased. This issue could be reduced by the use of a stylus or other pointing device.
\subsection{Issue 3: File Handling}
As evidence volumes increase the issue of locating the correct file increase. Paper based trials use an extensive indexing system to keep track of all evidence documents in the many volumes of paper documents. It was observed that often court participant had difficulty finding the correct place within lengthy documents when directed to related material. This often resulted in wasted time while the relevant text was located.\\
\textit{iPad app as a solution}\\
The iPad applications offer the ability create file structures, and have search functionality that allow the user to easily navigate to the correct location. There is no need for the user to recall the file type, or the file placement within the file structure.
\\
\textit{Test observations:}\\
During the usability application analysis test candidates selected the correct file on most occasions, but were only using a data pack of ten files. Some candidates commented that they found it difficult to find the correct files. The applications often stored all files in the one location with no folder structure to aid in sorting them. Exhibit A does use a folder structure for the sorting of files. In cases involving large numbers of files this issue will be greatly increased, and the importance of a folder structure will be magnified.\\


\subsection{Issue 4: Evidence Annotation}
Annotating of evidence is a common activity by all members of the judiciary. It was often observed that defence, prosecution, judges , and magistrates were annotation various documents and images during proceedings. These annotations were "private" and could not be shared with other court participants in order to highlight a particular point on the paper documents.
\\
\textit{iPad app as a solution}\\
All of the iPad applications offered the functionality to highlight, underline, redact, write notes, freehand draw on evidence files, both documents and images. The ability to remove a one or all annotations was also available.
\\
\textit{Test observations:}\\
Candidates often experienced difficulties while attempting to annotate both documents and images in both the usability application analysis and the moot court.\\
The issues experienced by candidates primarily related to to the lack of accuracy when interacting with the interface with their fingers. As an example, when attempting to underline a particular passage in a document candidates were observed to place the underline thorough the text, not under it. If the underline tool was freehand the line would tend to wonder across the text of the document. Highlighting and redact tool also suffered from a similar accuracy issue.\\
The issue of onscreen annotation was significantly increased when the candidate was attempting to use onscreen hand gestures to navigate for scrolling or zooming within a document or image. The annotation tool did not automatically exit, and if the candidate forgot to exit the annotation tool before attempting to navigate erroneous annotations  would be generated.\\
The saving of annotations on documents and images was also identified as an issue. Some applications saved the annotations by creating a screenshot of the annotated document or image and saving it as a separate .jpg file, showing only what what currently visible on the screen. This method of saving resulted in any annotation that was not visible on the screen at the time of saving being lost.
None of the applications tested had the ability to hyperlink between or within files. Candidates needed to close one file before navigating to and opening another, a time consuming task.


\subsection{Issue 5: Evidence Display Manipulation}
The way in which evidence could be displayed was also observed to be an issue. Hard copy evidence, both documents and images, were difficult to manipulate to allow the court participant to view in fine detail, or from a particular rotation. This was largely due to the hard copy documents or images being in large folders. The rotation of any images displayed using a video projector or screen was not possible.\\
\textit{iPad app as a solution}\\
Some tested applications had the ability to rotate documents and images, and also the ability to magnify areas within documents and images through a magnification pop-out tool.These tools did not in anyway alter the evidence, only the way that it was displayed at the time the tools were used.\\
\textit{Test observations}\\
Some of the tested applications did not have rotation resulting in some test candidates not being able to complete the image rotation task. Applications that did have these tools were still not ideal. One of the applications only allowed users to rotate the document or image in one direction, while the other while allowing multi-directional rotation only allowed 90 degree increments\\
The pop-out magnification tool, while quite useful, also presented a number of issues. The quality of the magnified image was quite low, particularly when magnifying low resolution images, often resulting in a very pixelated output. Several test candidates were observed having difficulties returning from the magnified image to the standard image, highlighting the need for a clear close operation on the tool. The tool also failed to show prior annotations or redactions on the magnified image, this is of particular concern, as it could easily lead to a mis-trial situation.

\subsection{Issue 6: Evidence Display}
Video evidence presentation in courtrooms is currently limited to the use of  mounted on trolleys or the walls of the courtroom, or data projection onto to the blank walls of the courtroom. In most cases this proved to be a highly ineffectual method of communicating the data to the members of the court. Often members of the court were unable to adequately see the screens due to their location within the courtroom, and members were often  seen removing reading glasses to view the video at some distance. In one case the judge was observed turning his back on half the court in order to view a video screen mounted on the side wall of the court.

\section{Closing Comments}


\end{document}
